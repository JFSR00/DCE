\documentclass[a4paper]{article}

\usepackage[utf8]{inputenc}
\usepackage[spanish]{babel}
\usepackage[T1]{fontenc} 
\usepackage[usenames]{color}

\newcommand\tab[1][1cm]{\hspace*{#1}}

\usepackage[a4paper,top=3cm,bottom=2cm,left=3cm,right=3cm,marginparwidth=1.75cm]{geometry}
\usepackage{lmodern}
\usepackage{vmargin}
\usepackage{xcolor}
\usepackage{listings}
\usepackage{graphicx}
\usepackage{afterpage}
\usepackage[hidelinks]{hyperref} 

\hypersetup{
	colorlinks=true,
	linkcolor=blue,
	filecolor=magenta,      
	urlcolor=cyan,
}

\usepackage{float}
\usepackage{amsmath}
\usepackage{subfigure}
\usepackage{amsmath}
\usepackage{parskip}
\usepackage{graphicx}
\usepackage{xcolor}
\usepackage{cancel}
\usepackage{amsmath}
\usepackage{amssymb}
\usepackage{amsfonts}
\usepackage{natbib}
\usepackage{ragged2e}
\usepackage[nottoc,notlot,notlof]{tocbibind}

\usepackage{color}
\definecolor{editorGray}{rgb}{0.95, 0.95, 0.95}
\definecolor{editorOcher}{rgb}{1, 0.5, 0} % #FF7F00 -> rgb(239, 169, 0)
\definecolor{editorGreen}{rgb}{0, 0.5, 0} % #007C00 -> rgb(0, 124, 0)
\usepackage{upquote}
\usepackage{listings}

% Fin de inicialización del documento ------------------------------------------------

\renewcommand*\contentsname{Índice} %Esto es para cambiar el título del índice

\begin{document}

\begin{titlepage}
	%---------------------------------------------------------------------------------------------------
	%-----------| TITULO |-----------------------------------------------------------------------------
	\centering
	\vspace{5cm}
	{\scshape\Huge Memoria proyecto robot \par}
	{\LARGE Emisión de sonidos con zumbador \par}
	\begin{figure}[H]
		\centering
		\includegraphics[width=13cm]{images/robot.jpg}
	\end{figure}
	{\bfseries\LARGE Diseño de Computadores Empotrados \par}
	{\Large Autor: \par}
	{\Large Juan Francisco Santos Relinque \par}
	{\Large Mayo 2021 \par}
	{\large Curso 2020/2021 \par}
	\begin{figure}[H]
		\raggedleft
		\includegraphics[width=2cm]{images/LogoV2_475x629.png}
	\end{figure}
\end{titlepage}
	
\newpage
%-------------------------------------------------------------------------------------------------------
%-----------| TABLA DE CONTENIDOS |---------------------------------------------------------------
%-------------------------------------------------------------------------------------------------------
\hypersetup{linkcolor=black}
\tableofcontents
\hypersetup{linkcolor=blue}

\newpage
\section{Introducción}

\section{Tecnologías a utilizar}

	\subsection{Descripción MCU}
	
	\subsection{Descripción robot}
	
		\subsubsection{Tabla de pines}
		
		\subsubsection{Dispositivos incluidos}

	\subsection{Lenguaje de desarrollo}
	
	\subsection{Entorno de desarrollo: Atmel Studio}
	
	\subsection{Entorno de programación: Avrdude.exe}
	
\section{Requisitos del diseño}

	\subsection{Señal acústica de aviso}
	
	\subsection{Otros requisitos ...}

\section{Test de los recursos del robot}

	\subsection{Descripción de los tests}
	
	\subsection{Resultados de los tests}

\section{Metodología del diseño}

	\subsection{Señal acústica de aviso}
	
		\subsubsection{Dispositivos electrónicos a utilizar}
		
		\subsubsection{Configuración de pines}
		
		\subsubsection{Periféricos del MCU}
		
		\subsubsection{Diseño del software}
		
	\subsection{Otras funcionalidades}

\section{Implementación}

	\subsection{Señal acústica de aviso}
	
		\subsubsection{Grafo de la FSM}
		
		\subsubsection{Librerías}
		
		\subsubsection{Verificación del diseño}
		
		\subsubsection{Descarga del fichero de configuración en MCU}
		
	\subsection{Otras funcionalidades}
	
	\subsection{Integración del software}
	
\bibliography{T01}

\appendix

\end{document}